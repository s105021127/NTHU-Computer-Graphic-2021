\pagenumbering{roman}
\setcounter{page}{3}
\prefacesection{Abstract}
        \begin{center}            \textbf{\Large{Automatic Sarcasm Detection through Emotion Reactions on Social Media Platforms}}
            \vskip 0.5cm
            \textbf{\Large{Po-Chen Kuo}}
        \end{center}
        Online social media users react to content in them based on context. Emotions or mood play a significant part of these reactions, which has filled these platforms with opinionated content. 
        Different approaches and applications to make better use of these data are continuously being developed.
        However, due to the nature of the data, the variety of platforms, and dynamic online user behavior, there are still many issues to be dealt with. It remains a challenge to properly obtain a reliable emotional status from a user prior to posting a comment. 
        This work introduces a methodology that explores semi-supervised multilingual emotion detection based on the overlap of Facebook reactions and textual data. With the resulting emotion detection system we evaluate the possibility of using emotions and user behavior features for the task of sarcasm detection.
        More than 1 million English and Chinese comments from over 62,000 public Facebook pages posts have been collected and processed, conducted experiments show acceptable performance metrics.


\clearpage
