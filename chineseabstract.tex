\pagenumbering{roman}
\setcounter{page}{2}
\prefacesection{摘要}
\begin{center}
    \textbf{\Large{從社群網路情緒反饋偵測諷刺情緒}}
    \vskip 0.5cm
    \textbf{\Large{郭柏辰}}
\end{center}
在社群媒體上,使用者們會根據上下文對內容做出反應。而在這些回應中,情緒與心情扮演了一個非常重要的腳色,這也讓這些社群網路平台上充滿了表達個人意見與看法的資料。為了更加有效的利用這些資料,許多不同的方法與應用正在不斷的發展。然而,由於資料的性質、平台的多樣性與使用者行為的變化,還有許多的課題必須去克服,在使用者發表評論之前獲得可靠的情緒狀態仍然是一個挑戰。本研究提出了一種基於Facebook Reactions與文本資料交集的半監督式多語言情感檢測方法,利用由此情感檢測方法所產生的成果,評估使用情感和用戶行為特徵來進行諷刺檢測的可能性。超過100萬份中英文評論與62,000多個公開的Facebook紛絲團文章被收集並處理,透過實驗得執行與成果顯示,使用情感和用戶行為特徵來進行諷刺檢測是能達到一定程度的性能衡量標準。

\clearpage
